\section{Einleitung}
Mit der zunehmenden Globalisierung steht heute praktisch jedes produzierende Unternehmen 
im internationalen Wettbewerb. Der Markt fordert hohe Flexibilität und Lieferfähigkeit, 
wobei die Kosten immer weiter sinken sollen, um wettbewerbsfähig zu bleiben.
Die japanische Firma Toyota hat diese Problematik für sich bereits in den 1950er
Jahren erkannt und eigene Lösungen dafür gesucht. So ist schliesslich im Laufe der
folgendenden Jahrzehnte das Toyota-Produktionssystem (TPS) entwickelt worden, dessen 
wesentliche Bestandteile heute auch unter den Begriffen  \emph{Just-in-Time} oder  
 \emph{Lean-Production} bekannt sind. Ein wichtiger Bestandteil des TPS ist das
Kanban-System, das vor allem von Taiichi Ohno entwickelt wurde.
\footnote{vgl. \cite{Ohno2013TPS}}\\
Die vorliegende Arbeit gibt eine Einführung in die wesentlichen Methoden, 
Werkzeuge und Prinzipien des Kanban-Systems und wie mit Hilfe von Kanban eine
Just-in-Time-Produktion umgesetzt werden kann. Im zweiten Abschnitt wird hierzu
die Geschichte und Entwicklung von Kanban vorgestellt.
Im Abschnitt drei wird aufgezeigt, wie Kanban im produzierenden Betrieb 
eingeführt werden kann und wie dadurch die Wettbewerbsfähigkeit gesteigert wird. 
Eng in Verbindung mit Kanban steht der kontinuierliche Verbesserungsprozess 
(KVP), auch Kaizen genannt, was im vierten Abschnitt betrachtet wird.
Im letzten Abschnitt wird kritisch auf die Risiken einer Kanban-Einführung
eingegangen und ein Ausblick auf die zu erwartende künftige Entwicklung gegeben.\\
Für die Erstellung dieser Arbeit wurde auf relevante Fachliteratur zurückgegriffen
und teilweise auch im Internet recherchiert.


\section{Geschichte und Entwicklung von Kanban}

Das Toyota-Produktionssystem entand im Japan der Nachkriegszeit 
aus der wirtschaftlichen Notwendigkeit heraus. 
Die Produktivität eines japanischen Arbeiters betrug zu der Zeit nur einen Bruchteil der Produktivität eines Amerikaners. 
Der damalige Präsident der Toyota Motor Company Kiichiro Toyoda (1894-1952) gab daher das Ziel vor, 
die US-amerikanische Automobilindustrie innerhalb von drei Jahren  einzuholen. \footnote{vgl. \cite{Ohno2013TPS} S.36}\\
Toyotas Produktionsleiter Taiichi Ohno war der Ansicht, man müsse alle Arten von 
Verschwendung von Material und Zeit und alle unproduktiven Tätigkeiten 
beseitigen, um den Rückstand aufzuholen. Als eine Hauptursache der Ineffizienz 
und Verschwendung identifizierte man bei Toyota die Überproduktion bzw. die Produktion auf 
Halde und damit verbunden übermässige Lagerhaltung. Dadurch wird Kapital gebunden, 
die Duchlaufzeiten erhöhen sich, durch Korrosion und häufigen Transport verschlechtert
sich die Qualität der Zwischenerzeugnisse und möglicherweise müssen soger zu viel
produzierte Teile weggeworfen werden. Bedingt durch die räumliche Enge stellt die 
Lagerhaltung in Japan auch prinzipiell ein höheren Kostenfaktor dar als beispielsweise in der USA.
Künftig sollte bei Toyota die Fließproduktion \emph{Just-in-Time} erfolgen. 
Das bedeuted, dass die für die Produktion benötigten Teile zur rechten Zeit und 
nur in der benötigten Menge am Fließband ankommen.
Auf diese Weise sollte der Lagerbestand auf ein Minimum reduziert werden.
Mit den herkömmlichen Verfahren zur Produktionsplanung vom Rohstoff 
bis zum Endprodukt war das nicht zu schaffen.
Stattdessen betrachtete Ohno den Materialfluss vom Ende her, also in entgegengesetzter Richtung.
Eine Produktionsstufe entnimmt sich die benötigten Teile von der vorgelagerten Stufe oder dem Zwischenlager,
die vorgelagerte Stufe produziert darauf hin die entnommene Menge an Teilen nach.
So entsteht ein sich selbst regelndes System von Produzent und Verbraucher.
Für die Informationsübertragung in diesem Regelkreis dienen die Kanban-Karten.

Das Wort \emph{Kanban} besteht aus den zwei Zeichen {\CN 看} ( \emph{kan}=sehen)
 und {\CN 板} (\emph{ban}=Tafel, Brett) und läßt sich etwa mit Sichttafel, 
 Aushängeschild oder auch Pendelkarte übersetzen.

Im Produktionsprozess steht jede Kanban-Karte für einen Behälter einer bestimmten Grösse, der eine festgelegt Anzahl von Bauteilen enthält.
Die Anzahl von Kanban-Karten für ein Bauteil oder eine Bauteilgruppe ist begrenzt, auf diese Weise soll verhindert werden, dass zu viel auf Lager produziert wird.
Der gesamte Produktionsprozess wird nun betrachtet als eine Aneinanderreihung von Quellen und Senken von Produktionsgütern, mit kleinen Zwischenlagern als Puffer.
Eine Senke nimmt sich einen Behälter aus dem Zwischenlager (Pull-Prinzip), verarbeitet alle Teile darin und füllt selbst als Quelle das nachgelagerte Zwischenlager.
Hat das nachfolgende Zwischenlager einen bestimmten Höchststand überschritten, \emph{darf nicht} weiter produziert werden.
Wird dagegen ein bestimmter Mindeststand unterschritten, so \emph{muss} wieder Nachschub produziert werden.
Auf diese Weise werden Probleme oder Engpässe schnell sichtbar und es können entsprechende Gegenmaßnahmen unternommen werden.
Andererseits können durch die mehrstufigen Zwischenlager Schwankungen bei Nachfrage, Zulieferung oder Personalstärke in gewissen Grenzen ausgeglichen werden.

Die Zwischenlager der einzelnen Produktionsstufen können auf einer \emph{Plantafel} (auch Kanban-Tafel) visualisiert werden, welche an zentraler Stelle für alle Beteiligten gut sichtbar platziert wird.
Für jede Produktionsstufe gibt es auf der Plantafel eine Spalte oder Zeile mit festen Plätzen für die Kanban-Karten.
Die Karten der leeren Behälter werden hier für jeden sichtbar plaziert, so dass auf einen Blick der Bestand der Zwischenlager erkennbar wird.


\section{Einführung von Kanban}
\subsection{Problemstellung}
Motivation: \\
- hoher Aufwand herkömmlicher PPS-Systeme\\
- Flexibilität und Lieferfähigkeit erhöhen\\
- Übereinstimmung des tatsächlichem Bestandes mit dem Bestand in der IT\\
- Reduzierung von Umlaufvermögen und Durchlaufzeiten\\
- Minimierung von Verlusten durch Verschwendung und Ausschuss\\
\subsection{Kanban-Fähigkeit}
\subsection{Auswahl der Regelkreise}
\subsection{Berechnung der Kanban-Grössen}

erfolgt schrittweise\\
Überprüfung der Kanbanfähigkeit (ABC-Teile, XY-Teile)\\
Glätten der Produktion, Verkleinerung der Losgrössen, Standardisierung der Teile (Takeda 2015, S. 7)\\
zuerst in einem Teilbereich, ein Team von ca 10 Personen organisiert sich eigenständig, Dauer 6-12 Monate\\
Höhe Verantwortung der einzelnen Mitarbeiter\\
Kanban-Verantwortlicher prüft und korrigiert regelmäßig die Mengen.\\

Mit grösserer Anzahl von Kanban beginnen, schrittweise reduzieren (Geiger et al 2011)\\
Schrittweise Verbesserungen vornehmen, aus Fehlern lernen.\\
Nach und nach auf andere Produktionsbereiche ausdehnen\\
Schließlich Einkauf und Lieferanten einbeziehen.\\

\subsection{Auswahl der Kanban-Hilfsmittel}
Produktions-Kanban, Transport-Kanban\\
Kanban-Karten\\
Kanban-Tafel\\
Behälter\\
Transportwagen\\
Stellflächen\\
-Karten, Tafel, Behälter, Stellflächen, Signallampen\\
-Gitterboxen, Europaletten, Kartonagen...\\

\subsection{Erfassung von Daten}
-Zur Kontrolle, Erstellung von Metriken, für PPS-System\\
-elektronische Systeme: Barcode, QR-Code, RFID-Etiketten (Funk)\\

\section{Kaizen: kontinuierliche Verbesserung}
- Permanentes überprüfen auf Optimierungspotential.\\
- Für alle sichtbare Visualisierung der Kennzahlen aus den Bereichen Mitarbeiter, Bestände, Kunden, Qualität, Sicherheit, Rüstzeiten\\
- Alle Mitarbeiter in KVP einbeziehen.\\
- Jeder kann Vorschläge machen.\\
- japanische Sichtweise: der Einzelne ist Teil des Ganzen\\
- Jeder soll Störungen und Fehler melden (bei Toyoto das gesamte Band anhalten).\\
- Dem Fehler auf den Grund gehen, 5 mal warum fragen, um das eigentliche Problem zu finden und zu beheben\\
- Fehler sollten immer zu Verbesserungen führen.\\
- Bsp: Lieferant wegen Unwetter verzögert: Sicherheitsbestand erhöhen, Dualsourcing einführen\\

\section{Risiken des Kanban-Systems}
Störung durch äußere Einflüsse, z.B. Streik, Unwetter, Vulkanausbruch, Flutkatastrophe, Unfälle.\\
-> Fokus auf Risikomanagement und Verbesserungen durch Kaizen\\

\section{Fazit und Ausblick}
Kanban ist 
