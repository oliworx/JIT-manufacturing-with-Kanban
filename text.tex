\section{Einleitung}
Mit der zunehmenden Globalisierung und ...

Die vorliegende Arbeit gibt einen Überblick über ....
Dafür wurde vor allem auf den Internet-Seiten ...

Im Abschnitt zwei der Arbeit wird ein historischer Überblick gegeben 
werden, der die Entwicklung ... aufzeigt, er endet mit einer 
Bestandsaufnahme der Verwendung und Verbreitung von Kanban im Jahr 2016. 
Der dritten Abschnitt soll einige der aktuellen Probleme und 
Herausforderungen aufzeigen, die im Zusammenhang mit Kanban stehen. 
Im letzten Abschnitt wird ein Fazit gezogen und ein Ausblick zu dem Thema gegeben.

\section{Geschichte von Kanban}

Kanban wurde als Methode der Produktionsplanung und -steuerung zuerst in der japanischen Automobil-Industrie der Kriegs- und Nachkriegszeit entwickelt und eingeführt.
Die Firma Toyota versuchte zu der Zeit, den Abstand zu den amerikanischen Wettbewerbern zu verringern.
Hierzu sollten die Kosten der Produktion verringert und die Qualität der Produkte verbessert werden.
Wegen den damals sehr begrenzten Ressourcen galt es jede Art von Verschwendung zu vermeiden. Dies gilt nicht nur für Rohstoffe, sondern auch für Zeit, Personal, Kapital.

Die seit dieser Zeit entwickelten Methoden sind heute auch als \emph{Lean Production}, \emph{just in Time} und \emph{Kontinuierlicher Verbesserungsprozess} (KVP)bekannt.

Das Wort \emph{Kanban} besteht aus den zwei Zeichen {\CN 看} ( \emph{kan}=sehen) und {\CN 板} (\emph{ban}=Tafel, Brett) und läßt sich etwa mit Sichttafel, Aushängeschild oder auch Pendelkarte übersetzen.
Diese Kanban-Karten sind ein zentraler Bestandteil im Kanban-Prozess.
Im industriellen Produktionsprozess steht jede Kanban-Karte für einen Behälter einer bestimmten Grösse, der eine festgelegt Anzahl von Bauteilen enthält.
Die Anzahl von Kanban-Karten für ein Bauteil oder eine Bauteilgruppe ist begrenzt, auf diese Weise soll verhindert werden, dass zu viel auf Lager produziert wird, denn große Zwischenlager binden Ressourcen.
Der gesamte Produktionsprozess wird betrachtet als eine Aneinanderreihung von Quellen und Senken von Produktionsgütern, mit Zwischenlagern als Puffer.
Eine Senke nimmt sich einen Behälter aus dem Zwischenlager (Pull-Prinzip), verarbeitet alle Teile darin und füllt (als Quelle) das nachgelagerte Zwischenlager.
Hat das nachfolgende Zwischenlager einen bestimmten Höchststand überschritten, \emph{darf nicht} weiter produziert werden.
Wird dagegen ein bestimmter Mindeststand unterschritten, so \emph{muss} wieder Nachschub produziert werden.
Auf diese Weise werden Probleme oder Engpässe schnell sichtbar und es können entsprechende Gegenmaßnahmen unternommen werden.
Andererseits können durch die mehrstufigen Zwischenlager Schwankungen bei Nachfrage, Zulieferung oder Personalstärke in gewissen Grenzen ausgeglichen werden.

Die Zwischenlager der einzelnen Produktionsstufen werden auf der \emph{Plantafel} (auch Kanban-Tafel) visualisiert, welche an zentraler Stelle für alle Beteiligten gut sichtbar platziert wird.
Für jede Produktionsstufe gibt es auf der Plantafel eine Spalte oder Zeile mit festen Plätzen für die Kanban-Karten.
Die Karten der leeren Behälter werden hier für jeden sichtbar plaziert, so dass auf einen Blick der Bestand der Zwischenlager erkennbar wird.

8,25
\subsection{Vorgeschichte}

\subsubsection{Baudot-Code}
Die Geschichte ...

\section{Fazit und Ausblick}
Kanban ist ...
