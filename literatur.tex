% Literaturliste soll im Inhaltsverzeichnis auftauchen
\newpage
\addcontentsline{toc}{section}{Literatur}
  \begin{thebibliography}{}

%  Monographien:
%1.  Familienname des Verfassers
%2.  Vorname des Verfassers, vorzugsweise abgekürzt
%3.  Erscheinungsjahr in Klammern
%4.  vollständiger Titel des Werks
%5.  Auflage (Aufl.), wenn es sich nicht um die erste Auflage handelt
%6.  Erscheinungsort, der grundsätzlich dem Verlagsort entspricht (mehr als 3 Orte sind nicht zu nennen; stattdessen: u.a.) und / oder Verlag
%7.  Erscheinungsjahr
%8.  Handelt es sich bei der Literaturquelle um eine Dissertation, ist vor 5. die Abkürzung „Diss.“ Einzufügen

% Beispiele:
% Scholz, C. (2000): Personalmanagement, 5. Aufl., München 2000
% Lüdenbach, N., Hoffmann, W.-D. (2003): Haufe IAS-Kommentar, Freiburg 2003
% Ruhnke, K. (1995): Konzernbuchführung, Diss., Düsseldorf 1995

% Zeitschriftenartikel oder Zeitungsartikel:
% 1.  Familienname des Verfassers
% 2.  Vorname des Verfassers, vorzugsweise abgekürzt
% 3.  Erscheinungsjahr in Klammern
% 4. vollständiger Titel des Artikels
% 5. nach dem Titel wird der Zusatz „in:“ eingefügt
% 6. Name der Zeitung in abgekürzter oder ausgeschriebener Form mit Erscheinungsjahr
% 7. Jahrgang (Jg.)
% 8. ggf.Heftnummer (Nr.),wenn keine durchgängige Paginierung(Durchnummerierung) erfolgt
% 9. Seitenzahl („S. ...“) bzw. Spaltenzahl („Sp. ...“), und zwar Anfang bis Ende
%
% Beispiele:
% Herzberg, F. (1968): Was Mitarbeiter in Schwung bringt, in: Harvard Business Manager, 
%   o.J., 2003, Nr. 4, S. 50-62
% Kümpel, Th. (2002): Integration von internem und externem Rechnungswesen bei der Bewertung
%   erfolgversprechender langfristiger Fertigungsaufträge, in: Der Betrieb 2002, 55. Jg., S. 905-910

	% Hauptquelle
	\bibitem[Geiger, G., Hering, E., Kummer, R. (2011)]{Geiger2011Kanban} {\sl Kanban}, 3. Auflage, Carl Hanser Verlag, München, 2011

	\bibitem[Lotter, B., Wiendahl, H.-P. (2013)]{lotter2013montage} {\sl Montage in der industriellen Produktion: Ein Handbuch für die Praxis}, 2. Auflage, Springer-Verlag, Berlin Heidelberg, 2013
	
	\bibitem[Ohno, T. (2013)]{Ohno2013TPS} {\sl Das Toyota-Produktionssystem}, 3. Auflage, Campus-Verlag, Frankfurt, 2013

	% Kapitel Kanban ab s. 177 ff
	\bibitem[Takeda, H. (2012)]{Takeda2012SPS} {\sl Das synchrone Produktionssystem: Just-in-time für das ganze Unternehmen}, 7. Auflage, Vahlen-Verlag, München, 2012

	% IT-gestützes Kanban 141ff
	\bibitem[Weber, R. (2014)]{Weber2014KE} {\sl Kanban-Einführung}, 8. Auflage, expert-Verlag, Renningen, 2014
	
	
	\textbf{Internetquellen:}
% 1. Nachname, ggf. Internetanbieter,
% 2. Vorname Jahr:
% 3. Titel.
% 4. URL: http://(Internetadresse),
% 5. Abruf am (Datum).
% Beispiel:
% Teltarif (2007): Rufnummerportierung im Mobilfunk. URL: http://www.teltarif.de/i/portierung.html, Abruf am 7.11.2007

	% Definition Just in Time
	\bibitem[Gabler Wirtschaftslexikon]{GablerJIT} {\sl Stichwort: Just in Time (JIT)} URL: \url{http://wirtschaftslexikon.gabler.de/Archiv/57306/just-in-time-jit-v10.html}, Abruf am 8.5.2016

	% Bei Problemen 5 mal nach dem Warum fragen.
	\bibitem[Taiichi Ohno, Toyota Global Site, 2006]{Toyota}{\sl Ask 'why' five times about every matter.} URL: \url{	http://www.toyota-global.com/company/toyota_traditions/quality/mar_apr_2006.html}, Abruf am 9.5.2016
	
	% Cluster der Autoindustrie
	\bibitem[The Economist, 2009]{Economist2009Ohno}{\sl Taiichi Ohno} URL: \url{http://www.economist.com/node/13941150}, Abruf am 9.5.2016
	
	% Bild optimale Losgrösse
	\bibitem[The Economist, 2009]{Economist2009Ohno}{\sl Taiichi Ohno} URL: \url{http://www.daswirtschaftslexikon.com/d/losgr%C3%B6sse/losgr%C3%B6sse.htm}, Abruf am 9.5.2016
	
	 
  \end{thebibliography}
