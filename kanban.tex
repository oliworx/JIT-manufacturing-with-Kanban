% Kompilieren mit TeXworks/ XeLaTeX
\documentclass[12pt,german,a4paper]{scrartcl}

% Abstand des Textes vom Rand sollte sich an folgenden Richtwerten orientieren: 
%     a.  Oben: 3 cm, Seitenzahl steht 1,5 cm von der oberen Papierkante 
%     b.  Unten: 2 cm (letzte Zeile der letzten Fußnote bis untere Papierkante) 
%     c.  Links: 4 cm 
%     d.  Rechts: 2 cm  
%
% Mit  Ausnahme  des  Titelblatts  sind  alle  Seiten  der  Arbeit  fortlaufend  innerhalb 
% der  oberen  Kopfzeile  zu  nummerieren.  Dabei  sind,  beginnend  mit  der  ersten 
% Seite  des  Inhaltsverzeichnisses, sämtliche  Verzeichnisse  vor  dem  Textteil  mit 
% römischen  Ziffern  fortlaufend  durchzunummerieren.  Beginnend  mit  der  ersten 
% Seite des Textteils werden alle Seiten – mit Ausnahme der ehrenwörtlichen Er-
% klärung – mit arabischen Ziffern fortlaufend nummeriert (neu beginnend mit „1“). 

\usepackage[left=4cm,right=2cm,top=3cm,bottom=2cm]{geometry}

% Seitenzahlen oben zentriert
\usepackage{scrpage2} 
\clearscrheadfoot 
\chead[\pagemark]{ \pagemark} 
\pagestyle{scrheadings} 

% deutsche Silbentrennung
\usepackage[ngerman]{babel}

% wegen deutschen Umlauten
\usepackage[utf8]{inputenc}

% 1,5-zeilig
\usepackage{setspace}
\onehalfspacing

% chinesische Zeichen
\usepackage{xltxtra} 
\newfontfamily\CN{SimSun}

% Abkürzungsverzeichnis
\usepackage{acronym}

\usepackage{fontspec}
\usepackage{hyperref}
\usepackage{enumitem}
% \usepackage{color}

 % kein Abstand bei Listen zwischen den Items
\setlist{nolistsep,partopsep=0pt}

% keine Einrückung bei Absätzen
\setlength{\parindent}{0pt} 

\addtokomafont{sectioning}{\rmfamily}

% hier beginnt das Dokument
\begin{document}

% Festlegung Art der Zitierung - Havardmethode: Abkuerzung Autor + Jahr
\bibliographystyle{alphadin}


% Die Titelseite
\begin{titlepage}
% Name der Hochschule und des Fachbereiches
\titlehead{\large FOM Hochschule für Oekonomie \& Management Essen\\ 
Standort München\\
\normalsize Berufsbegleitender Studiengang zum B.Sc.\ Wirtschaftsinformatik}
% Was ist das hier überhaupt? :)
\subject{\vspace{1cm}Seminararbeit}
% Thema der Arbeit
\title{Implementierung der Just-in-time-Produktion mittels Kanban}
% Autor mit Adresse
\author{
 {\normalsize Eingereicht von:}
   \\Oliver Kurmis\\
  {\normalsize Matrikel-Nr: 328091}\\
  {\normalsize Betreuer: Prof. Dr. Kemal Orak}
}
% Datum mit Semesterangabe
\date{{\normalsize Abgegeben am:}\\15. Mai 2016\\\bigskip{\normalsize Erarbeitet im:}\\7. Semester}

\end{titlepage}

\maketitle 

% keine Seitenzahl auf der Titelseite anzeigen
\thispagestyle{empty}

%Verzeichnisse

% vor dem Hauptteil römische Seitenzahlen verwenden
\pagenumbering{Roman}
\newpage
% dies ist die erste Seite bezüglich der Seitenzahlen
\setcounter{page}{1}

% Inhaltsverzeichnis
\tableofcontents
\newpage

% Abkürzungsverzeichnis
\phantomsection \addcontentsline{toc}{section}{Abkürzungsverzeichnis}
\section*{Abkürzungsverzeichnis}
\begin{acronym}[TPS] % für die Ausrichtung die längste Abk. hier in eckigen Klammern
 \acro{JIT}{Just-In-Time: rechtzeitig, fertigungssynchron, bedarfsorientiert} 
 \acro{PPS}{Produktionsplanung und -Steuerung}
 \acro{TPS}{Toyota-Produktions-System}

\end{acronym}

% ggf. Abbildungsverzeichnis auf eine neue Seite
 \newpage 
 \addcontentsline{toc}{section}{Abbildungsverzeichnis}
 \listoffigures

% ggf. Tabellenverzeichnis
% \addcontentsline{toc}{section}{Tabellenverzeichnis}
% \listoftables

\newpage
\pagenumbering{arabic}

\section{Einleitung}
Mit der zunehmenden Globalisierung und ...\\
Motivation: \\
- hoher Aufwand herkömmlicher PPS-Systeme\\
- Flexibilität und Lieferfähigkeit erhöhen\\
- Übereinstimmung des tatsächlichem Bestandes mit dem Bestand in der IT\\
- Reduzierung von Umlaufvermögen und Durchlaufzeiten\\
- Minimierung von Verlusten durch Verschwendung und Ausschuss\\

Die vorliegende Arbeit gibt einen Überblick über ....\\
Dafür wurde vor allem auf den Internet-Seiten ...\\

Im Abschnitt zwei der Arbeit wird ein historischer Überblick gegeben 
werden, der die Entwicklung ... aufzeigt, er endet mit einer \\
Bestandsaufnahme der Verwendung und Verbreitung von Kanban im Jahr 2016. \\
Der dritten Abschnitt soll einige der aktuellen Probleme und 
Herausforderungen aufzeigen, die im Zusammenhang mit Kanban stehen. \\
Im letzten Abschnitt wird ein Fazit gezogen und ein Ausblick zu dem Thema gegeben.

\section{Geschichte und Entwicklung von Kanban}

Kanban wurde als Methode der Produktionsplanung und -steuerung zuerst in der 
japanischen Automobil-Industrie der Kriegs- und Nachkriegszeit entwickelt und 
eingeführt.

Die Firma Toyota versuchte zu der Zeit, den Abstand zu den amerikanischen 
Wettbewerbern zu verringern.

Hierzu sollten die Kosten der Produktion verringert und die Qualität der 
Produkte verbessert werden.

Wegen den damals sehr begrenzten Ressourcen galt es jede Art von Verschwendung 
zu vermeiden. Dies gilt nicht nur für Rohstoffe, sondern auch für Zeit, 
Personal, Kapital.

Die seit dieser Zeit entwickelten Methoden sind heute auch als 
\emph{Lean Production}, \emph{just in Time} und 
\emph{Kontinuierlicher Verbesserungsprozess} (KVP)bekannt.

Das Wort \emph{Kanban} besteht aus den zwei Zeichen {\CN 看} ( \emph{kan}=sehen)
 und {\CN 板} (\emph{ban}=Tafel, Brett) und läßt sich etwa mit Sichttafel, 
 Aushängeschild oder auch Pendelkarte übersetzen.
Diese Kanban-Karten sind ein zentraler Bestandteil im Kanban-Prozess.

Im industriellen Produktionsprozess steht jede Kanban-Karte für einen Behälter einer bestimmten Grösse, der eine festgelegt Anzahl von Bauteilen enthält.
Die Anzahl von Kanban-Karten für ein Bauteil oder eine Bauteilgruppe ist begrenzt, auf diese Weise soll verhindert werden, dass zu viel auf Lager produziert wird, denn große Zwischenlager binden Ressourcen.
Der gesamte Produktionsprozess wird betrachtet als eine Aneinanderreihung von Quellen und Senken von Produktionsgütern, mit Zwischenlagern als Puffer.
Eine Senke nimmt sich einen Behälter aus dem Zwischenlager (Pull-Prinzip), verarbeitet alle Teile darin und füllt (als Quelle) das nachgelagerte Zwischenlager.
Hat das nachfolgende Zwischenlager einen bestimmten Höchststand überschritten, \emph{darf nicht} weiter produziert werden.
Wird dagegen ein bestimmter Mindeststand unterschritten, so \emph{muss} wieder Nachschub produziert werden.
Auf diese Weise werden Probleme oder Engpässe schnell sichtbar und es können entsprechende Gegenmaßnahmen unternommen werden.
Andererseits können durch die mehrstufigen Zwischenlager Schwankungen bei Nachfrage, Zulieferung oder Personalstärke in gewissen Grenzen ausgeglichen werden.

Die Zwischenlager der einzelnen Produktionsstufen werden auf der \emph{Plantafel} (auch Kanban-Tafel) visualisiert, welche an zentraler Stelle für alle Beteiligten gut sichtbar platziert wird.
Für jede Produktionsstufe gibt es auf der Plantafel eine Spalte oder Zeile mit festen Plätzen für die Kanban-Karten.
Die Karten der leeren Behälter werden hier für jeden sichtbar plaziert, so dass auf einen Blick der Bestand der Zwischenlager erkennbar wird.


\subsection{Einführung von Kanban}
erfolgt schrittweise\\
Überprüfung der Kanbanfähigkeit (ABC-Teile, XY-Teile)\\
Glätten der Produktion, Verkleinerung der Losgrössen, Standardisierung der Teile (Takeda 2015, S. 7)\\
zuerst in einem Teilbereich, ein Team von ca 10 Personen organisiert sich eigenständig, Dauer 6-12 Monate\\
Höhe Verantwortung der einzelnen Mitarbeiter\\
Kanban-Verantwortlicher prüft und korrigiert regelmäßig die Mengen.\\

Mit grösserer Anzahl von Kanban beginnen, schrittweise reduzieren (Geiger et al 2011)\\
Schrittweise Verbesserungen vornehmen, aus Fehlern lernen.\\
Nach und nach auf andere Produktionsbereiche ausdehnen\\
Schließlich Einkauf und Lieferanten einbeziehen.\\

\subsubsection{Auswahl der Kanban-Hilfsmittel}
-Karten, Tafel, Behälter, Stellflächen, Signallampen\\
-Gitterboxen, Europaletten, Kartonagen...\\

\subsection{Erfassung von Daten}
-Zur Kontrolle, Erstellung von Metriken, für PPS-System\\
-elektronische Systeme: Barcode, QR-Code, RFID-Etiketten (Funk)\\

\subsection{Kaizen: kontinuierliche Verbesserung}
- Permanentes überprüfen auf Optimierungspotential.\\
- Für alle sichtbare Visualisierung der Kennzahlen aus den Bereichen Mitarbeiter, Bestände, Kunden, Qualität, Sicherheit, Rüstzeiten\\
- Alle Mitarbeiter in KVP einbeziehen.\\
- Jeder kann Vorschläge machen.\\
- japanische Sichtweise: der Einzelne ist Teil des Ganzen\\
- Jeder soll Störungen und Fehler melden (bei Toyoto das gesamte Band anhalten).\\
- Dem Fehler auf den Grund gehen, 5 mal warum fragen, um das eigentliche Problem zu finden und zu beheben\\
- Fehler sollten immer zu Verbesserungen führen.\\
- Bsp: Lieferant wegen Unwetter verzögert: Sicherheitsbestand erhöhen, Dualsourcing einführen\\

\subsection{Risiken und Nachteile von Kanban}
Störung durch äußere Einflüsse, z.B. Streik, Unwetter, Vulkanausbruch, Flutkatastrophe, Unfälle.\\
-> Fokus auf Risikomanagement und Verbesserungen durch Kaizen\\

\section{Fazit und Ausblick}
Kanban ist 

\input{anhang}
% Literaturliste soll im Inhaltsverzeichnis auftauchen
\newpage
\addcontentsline{toc}{section}{Literatur}
  \begin{thebibliography}{}

%  Monographien:
%1.  Familienname des Verfassers
%2.  Vorname des Verfassers, vorzugsweise abgekürzt
%3.  Erscheinungsjahr in Klammern
%4.  vollständiger Titel des Werks
%5.  Auflage (Aufl.), wenn es sich nicht um die erste Auflage handelt
%6.  Erscheinungsort, der grundsätzlich dem Verlagsort entspricht (mehr als 3 Orte sind nicht zu nennen; stattdessen: u.a.) und / oder Verlag
%7.  Erscheinungsjahr
%8.  Handelt es sich bei der Literaturquelle um eine Dissertation, ist vor 5. die Abkürzung „Diss.“ Einzufügen

% Beispiele:
% Scholz, C. (2000): Personalmanagement, 5. Aufl., München 2000
% Lüdenbach, N., Hoffmann, W.-D. (2003): Haufe IAS-Kommentar, Freiburg 2003
% Ruhnke, K. (1995): Konzernbuchführung, Diss., Düsseldorf 1995

	\bibitem[Becker, J.D. (1988)]{Becker88} {\sl Unicode 88} , Xerox Corporation, Palo Alto, CA, 1988
	
	\textbf{Internetquellen:}
	 
	\bibitem[Andrew West (2007)]{West2007} {\sl A Brief History of CJK-C} URL: \url{http://babelstone.blogspot.de/2007/06/brief-history-of-cjk-c.html}, Abruf am 31.1.2015

  \end{thebibliography}

\newpage

% keine Seitenzahl anzeigen
\thispagestyle{empty}

\section*{Ehrenwörtliche Erklärung}  
 
Hiermit  versichere  ich, dass die vorliegende Arbeit von  mir  selbstständig  und 
ohne unerlaubte Hilfe angefertigt worden ist, insbesondere dass ich alle Stellen, 
die wörtlich oder annähernd wörtlich aus Veröffentlichungen entnommen sind, 
durch Zitate als solche gekennzeichnet habe.
Weiterhin erkläre ich, dass die Arbeit in gleicher oder ähnlicher Form noch keiner 
anderen  Prüfungsbehörde  vorgelegen  hat.  Ich  erkläre  mich  damit  einverstanden, 
dass  die  Arbeit  der  Öffentlichkeit  zugänglich  gemacht wird. 
Ich erkläre mich damit einverstanden, dass die Digitalversion dieser 
Arbeit zwecks Plagiatsprüfung auf die Server externer Anbieter hoch geladen 
werden  darf.  Die  Plagiatsprüfung  stellt  keine Zurverfügungstellung  für  die 
Öffentlichkeit dar.
 
München, 15. Mai 2016

\includegraphics[width=.2\textwidth]{img/unterschrift.png}

Oliver Kurmis


\end{document}
